\chapter{组成及要求}

\section{正文}

正文一般包括绪论、论文主体、研究结论与展望等部分。

\subsection{绪论}

绪论应包括选题的背景和意义,国内外相关研究成果与进展述评,本论文所要解决的科学与
技术问题、所运用的主要理论和方法、基本思路和论文结构等。绪论应独立成章,用足够的
文字叙述,不与摘要雷同。要实事求是,不夸大也不弱化前人的工作和自己的工作。

\subsection{论文主体}

论文主体是正文的核心部分,占主要篇幅,它是将学习和研究过程中调查、观察和测试所获
得的材料和数据,经过思考判断、加工整理和分析研究,进而形成论点。依据学科专业及具
体选题,论文主体可以有不同的表现形式,可以按照章与节的结构表述,也可以按照“研究
背景与意义—研究方法与过程—研究结果与讨论”的表述形式组织论文。但主体内容必须实事
求是,客观诚实,准确完备,合乎逻辑,层次分明,简明可读。

\subsection{研究结论与展望}

研究结论是对整个论文主要成果的总结,不是正文中各章小结的简单重复,应准确、完整、
明确、精炼。应明确凝练出本研究的主要创新点,对论文的学术价值和应用价值等加以分析
和评价,说明本项研究的局限性或研究中尚难解决的问题,并提出今后进一步在本研究方向
进行研究工作的设想或建议。结论部分应严格区分本人研究成果与他人科研成果的界限。

\section{参考文献}

本着严谨求实的科学态度撰写论文,凡学位论文中有引用或参考、借鉴他人思想或成果之
处,均应按一定的引用规范,列于文末(通篇正文之后),参考文献部分应与正文的文献引
用一一对应,注重合理引用,严禁抄袭剽窃等学术不端行为。

\section{附录 (若有)}

主要列入正文内过分冗长的公式推导、供查读方便所需的辅助性数学工具或表格、数据图
表、程序全文及说明、调查问卷、实验说明等。

\section{致谢}

对给予各类资助、指导和协助完成研究工作,以及提供各种对论文工作有利条件的单位及个
人表示感谢。致谢应实事求是,切忌浮夸与庸俗之词。\\
致谢末尾应具日期,日期与论文封面一致。

\section{作者简历及攻读学位期间发表的学术论文与其他相关学术成果}

作者简历应包括从大学起到申请学位时的个人学习工作经历。\\
按学术论文发表的时间顺序,列出作者本人在攻读学位期间发表或已录用的学术论文清单
(著录格式同参考文献)。其他相关学术成果可以是申请的专利、获得的奖项及完成的项目
等代表本人学术成就的各类成果。